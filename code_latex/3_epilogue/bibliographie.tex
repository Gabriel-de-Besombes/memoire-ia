\newpage % Pour commencer la bibliographie sur une nouvelle page

% Ajoute une entrée "Bibliographie" dans la table des matières
\addcontentsline{toc}{chapter}{Bibliographie} 

\begin{thebibliography}{1}

% \bibitem{auteur_article}
% NOM Prénom. « Titre de l'article entre guillemets ». \textit{Titre de la revue en italique}, volume X, numéro Y, saison Année, p. 123-456.
% \bibitem{france_assureurs} 
% FRANCE ASSUREURS. \textit{L'assurance vie en 2024}. (23 septembre 2025). Consulté le 28 Octobre 2025, sur \url{https://www.franceassureurs.fr/nos-chiffres-cles/assurance-vie/lassurance-vie-en-2024/}

\bibitem{banque_de_france}
BANQUE DE FRANCE. \textit{Les taux monétaires directeurs}. (5 février 2026). Consulté le 18 février 2026, sur \url{https://www.banque-france.fr/fr/les-taux-monetaires-directeurs}



% \bibitem{clustering_book}
% DORNAIKA Fadi, HAMAD Denis, CONSTANTIN Joseph, TRONG HOANG Vinh. \textit{Advances in Data Clustering}. Lieu d'édition : Springer, 2024.

% \bibitem{goffard_guerrault}
% GOFFARD Pierre-Olivier, GUERRAULT Xavier. « Is it optimal to group policyholders by age, gender, and seniority for BEL computations based on model points? ». \textit{European Actuariel Journal}, volume 5, 17 Avril 2015, p. 165-180.


\end{thebibliography}